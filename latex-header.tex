%% ============================================================
%% CABLES.GL BOOK - CUSTOM STYLING
%% ============================================================
%% 
%% EASY-TO-CHANGE SETTINGS ARE BELOW
%% Modify these values to adjust typography throughout the book
%%
%% ============================================================

%% ------------------------------------------------------------
%% FONT SIZES AND LEADING (line spacing)
%% Format: \fontsize{SIZE}{LEADING} where LEADING = line height
%% ------------------------------------------------------------

%% Section 1 (Chapter titles) - largest
\newcommand{\SectionOneFontSize}{24}
\newcommand{\SectionOneLeading}{19}

%% Section 2 (Major sections)
\newcommand{\SectionTwoFontSize}{18}
\newcommand{\SectionTwoLeading}{15}

%% Section 3 (Subsections)
\newcommand{\SectionThreeFontSize}{14}
\newcommand{\SectionThreeLeading}{12}

%% Section 4 / Paragraph headers
\newcommand{\SectionFourFontSize}{12}
\newcommand{\SectionFourLeading}{11}

%% ------------------------------------------------------------
%% SECTION COLORS (RGB values 0-255)
%% ------------------------------------------------------------

%% Section 1 color (Chapter titles)
\newcommand{\SectionOneColorRed}{51}
\newcommand{\SectionOneColorGreen}{51}
\newcommand{\SectionOneColorBlue}{51}

%% Section 2 color (Major sections) - darker blue
\newcommand{\SectionTwoColorRed}{0}
\newcommand{\SectionTwoColorGreen}{40}
\newcommand{\SectionTwoColorBlue}{120}

%% Section 3 color (Subsections)
\newcommand{\SectionThreeColorRed}{51}
\newcommand{\SectionThreeColorGreen}{51}
\newcommand{\SectionThreeColorBlue}{51}

%% Section 4 color (Paragraph headers)
\newcommand{\SectionFourColorRed}{51}
\newcommand{\SectionFourColorGreen}{51}
\newcommand{\SectionFourColorBlue}{51}

%% Code block line spacing (1.0 = normal, 0.85 = 15% tighter)
\newcommand{\CodeBlockLineSpacing}{0.85}

%% Code block font size (in points)
\newcommand{\CodeBlockFontSize}{10.5}

%% ------------------------------------------------------------
%% PAGE HEADER AND FOOTER SETTINGS
%% ------------------------------------------------------------

%% Header height (in points)
\newcommand{\HeaderHeight}{12}

%% Footer height (20% more than header = HeaderHeight * 1.2)
\newcommand{\FooterHeight}{14.4}

%% ------------------------------------------------------------
%% IMAGE SETTINGS
%% ------------------------------------------------------------

%% General image width as fraction of line width (0.7 = 70%)
\newcommand{\ImageWidthFraction}{0.7}

%% Op image width (for the hundreds of operator images) - fraction of line width
\newcommand{\OpImageWidthFraction}{0.5}

%% Space REDUCTION after op images (negative = less space)
\newcommand{\OpImageSpaceAfter}{-16pt}

%% Ops section leading (line spacing for section 13 op entries)
\newcommand{\OpsSectionLeading}{0.85}

%% ============================================================
%% END OF EASY-TO-CHANGE SETTINGS
%% ============================================================


%% Basic packages
\usepackage{graphicx}
\usepackage{xcolor}
\usepackage{xstring}  % For string comparison in image handling
\usepackage{ragged2e} % For better left alignment
\usepackage{setspace} % For line spacing control
\usepackage{etoolbox} % For hooks and patches

%% Image handling - scale images and left-align
\usepackage{adjustbox}
\usepackage{float}
\makeatletter
\def\maxwidth{\ifdim\Gin@nat@width>\linewidth\linewidth\else\Gin@nat@width\fi}
\def\scaledwidth{\ImageWidthFraction\maxwidth}
\def\maxheight{\ifdim\Gin@nat@height>\textheight\textheight\else\Gin@nat@height\fi}
\makeatother
\setkeys{Gin}{width=\ImageWidthFraction\linewidth,height=\maxheight,keepaspectratio}

%% Force all images to be left-aligned (no centering)
\usepackage{caption}
\captionsetup{labelformat=empty,textformat=empty,justification=raggedright,singlelinecheck=false}

%% Completely override figure environment and centering to force left alignment
\makeatletter
% Override centering command globally to do left alignment
\renewcommand{\centering}{%
  \rightskip0pt plus 1fil
  \leftskip0pt
  \parfillskip0pt
  \raggedright
}
% Override center environment to be left-aligned
\renewenvironment{center}{%
  \par\noindent%
  \raggedright%
  \hspace{0pt}%
}{%
  \par%
}
% Completely replace figure environment - remove any text that might appear
\renewenvironment{figure}[1][htbp]{%
  \def\@captype{figure}%
  \par\vspace{\baselineskip}%
  \noindent%
  \raggedright%
  \hspace{0pt}%
}{%
  \par\vspace{\baselineskip}%
}
% Completely disable floatboxreset (which often centers)
\let\@floatboxreset\relax
\makeatother

%% Force left alignment for ALL includegraphics commands
%% Special handling for op images (in images/ops/ folder)
\makeatletter
\let\oldincludegraphics\includegraphics
\newcommand{\isopimage}[1]{%
  \IfSubStr{#1}{images/ops/}{\@firstoftwo}{\@secondoftwo}%
}
\renewcommand{\includegraphics}[2][]{%
  \par\noindent%
  \hspace{0pt}%
  \isopimage{#2}{%
    % Op image: use OpImageWidthFraction and reduced space after
    \oldincludegraphics[width=\OpImageWidthFraction\linewidth,keepaspectratio]{#2}%
    \vspace{\OpImageSpaceAfter}%
  }{%
    % Regular image: use standard settings
    \oldincludegraphics[#1]{#2}%
  }%
  \par%
}
\makeatother

%% TOC spacing
\usepackage{tocloft}
\setlength{\cftsecnumwidth}{2.5em}
\setlength{\cftsubsecnumwidth}{3.2em}
\setlength{\cftsubsubsecnumwidth}{4em}
\setlength{\cftparanumwidth}{4.8em}

%% Font fallbacks
\usepackage{fontspec}

%% Define colors using the variables above
\definecolor{sectiononecolor}{RGB}{\SectionOneColorRed,\SectionOneColorGreen,\SectionOneColorBlue}
\definecolor{sectiontwocolor}{RGB}{\SectionTwoColorRed,\SectionTwoColorGreen,\SectionTwoColorBlue}
\definecolor{sectionthreecolor}{RGB}{\SectionThreeColorRed,\SectionThreeColorGreen,\SectionThreeColorBlue}
\definecolor{sectionfourcolor}{RGB}{\SectionFourColorRed,\SectionFourColorGreen,\SectionFourColorBlue}
\definecolor{linkblue}{RGB}{0,102,204}

%% Custom header styles using the settings above
\usepackage{titlesec}

%% Section 1 - largest (chapter titles)
\titleformat{\section}
  {\fontsize{\SectionOneFontSize}{\SectionOneLeading}\bfseries\color{sectiononecolor}}
  {\thesection}{1em}{}

%% Section 2 (major sections)
\titleformat{\subsection}
  {\fontsize{\SectionTwoFontSize}{\SectionTwoLeading}\bfseries\color{sectiontwocolor}}
  {\thesubsection}{1em}{}

%% Section 3 (subsections)
\titleformat{\subsubsection}
  {\fontsize{\SectionThreeFontSize}{\SectionThreeLeading}\bfseries\color{sectionthreecolor}}
  {\thesubsubsection}{1em}{}

%% Apply reduced leading to ops sections (section 13 content)
%% Note: The \OpsSectionLeading variable is defined above and can be adjusted.
%% To apply it, you can manually add \setstretch{\OpsSectionLeading} in section 13 content
%% or use it in custom formatting. For now, we'll keep it as a variable for future use.

%% Section 4 / Paragraph - not numbered
\titleformat{\paragraph}
  {\fontsize{\SectionFourFontSize}{\SectionFourLeading}\bfseries\color{sectionfourcolor}}
  {}{0em}{}

%% Set section numbering depth to 3
\setcounter{secnumdepth}{3}

%% Code block styling
\usepackage{fancyvrb}
\usepackage[framemethod=tikz]{mdframed}

\definecolor{codebg}{RGB}{204,204,204}

\mdfdefinestyle{codeboxstyle}{
    backgroundcolor=codebg,
    linecolor=codebg,
    linewidth=0pt,
    innerleftmargin=16pt,
    innerrightmargin=16pt,
    innertopmargin=12pt,
    innerbottommargin=12pt,
    leftmargin=0pt,
    rightmargin=0pt,
    roundcorner=4pt,
    skipabove=\baselineskip,
    skipbelow=\baselineskip
}

%% Set code block line spacing and font size using the settings above
\DefineVerbatimEnvironment{Verbatim}{Verbatim}{
    baselinestretch=\CodeBlockLineSpacing,
    fontsize=\CodeBlockFontSize pt
}

\surroundwithmdframed[style=codeboxstyle]{Verbatim}
\surroundwithmdframed[style=codeboxstyle]{verbatim}

%% Page headers and footers
\usepackage{fancyhdr}
\pagestyle{fancy}
\fancyhf{} % Clear all headers and footers
\fancyfoot[R]{\thepage} % Right-aligned page number
\renewcommand{\headrulewidth}{0pt} % No header rule
\renewcommand{\footrulewidth}{0pt} % No footer rule

% Set header and footer heights with vertically centered page number
\setlength{\headheight}{\HeaderHeight pt}
\setlength{\footskip}{\FooterHeight pt}

% Vertically center page number in footer
\fancyfootoffset{0pt}
\renewcommand{\footruleskip}{0pt}

% Fix page 1 (plain style) to also have right-aligned page number
\fancypagestyle{plain}{%
  \fancyhf{} % Clear all headers and footers
  \fancyfoot[R]{\thepage} % Right-aligned page number
  \renewcommand{\headrulewidth}{0pt}
  \renewcommand{\footrulewidth}{0pt}
}

%% Link colors and final image alignment fixes
\AtBeginDocument{
  \hypersetup{
    colorlinks=true,
    linkcolor=sectiononecolor,
    urlcolor=linkblue,
    citecolor=sectiononecolor,
    filecolor=sectiononecolor
  }
  % Final override to ensure figures are left-aligned
  \makeatletter
  \renewcommand{\centering}{%
    \rightskip0pt plus 1fil
    \leftskip0pt
    \parfillskip0pt
    \raggedright
  }
  \renewenvironment{center}{%
    \par\noindent%
    \raggedright%
    \hspace{0pt}%
  }{%
    \par%
  }
  \makeatother
}
